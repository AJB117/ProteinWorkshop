\documentclass{article}

% if you need to pass options to natbib, use, e.g.:
%     \PassOptionsToPackage{numbers, compress}{natbib}
% before loading neurips_data_2023

% ready for submission
\usepackage{neurips_data_2023}

% to compile a preprint version, add the [preprint] option, e.g.:
%     \usepackage[preprint]{neurips_data_2023}
% This will indicate that the work is currently under review.

% to compile a camera-ready version, add the [final] option, e.g.:
%     \usepackage[final]{neurips_data_2023}

% to avoid loading the natbib package, add option nonatbib:
%    \usepackage[nonatbib]{neurips_data_2023}

% Submissions to the datasets and benchmarks are typically non anonymous,
% but anonymous submissions are allowed. If you feel that you must submit 
% anonymously, you can compile an anonymous version by adding the [anonymous] 
% option, e.g.:
%     \usepackage[anonymous]{neurips_data_2023}
% This will hide all author names.

\usepackage[utf8]{inputenc} % allow utf-8 input
\usepackage[T1]{fontenc}    % use 8-bit T1 fonts
\usepackage{hyperref}       % hyperlinks
\usepackage{url}            % simple URL typesetting
\usepackage{booktabs}       % professional-quality tables
\usepackage{amsfonts}       % blackboard math symbols
\usepackage{nicefrac}       % compact symbols for 1/2, etc.
\usepackage{microtype}      % microtypography
\usepackage{xcolor}         % colors
\usepackage{adjustbox}
\bibliographystyle{unsrtnat}
\setcitestyle{numbers,open={[},close={]},citesep={,}}

%\title{Evaluating Representation Learning on Protein Structures}
\title{Evaluating Representation Learning on the Protein Universe}

% The \author macro works with any number of authors. There are two commands
% used to separate the names and addresses of multiple authors: \And and \AND.
%
% Using \And between authors leaves it to LaTeX to determine where to break the
% lines. Using \AND forces a line break at that point. So, if LaTeX puts 3 of 4
% authors names on the first line, and the last on the second line, try using
% \AND instead of \And before the third author name.

\author{%
  David S.~Hippocampus%\thanks{Use footnote for providing further information
    %about author (webpage, alternative address)---\emph{not} for acknowledging
    %funding agencies.} \\
  Department of Computer Science\\
  Cranberry-Lemon University\\
  Pittsburgh, PA 15213 \\
  \texttt{hippo@cs.cranberry-lemon.edu} \\
  % examples of more authors
  % \And
  % Coauthor \\
  % Affiliation \\
  % Address \\
  % \texttt{email} \\
  % \AND
  % Coauthor \\
  % Affiliation \\
  % Address \\
  % \texttt{email} \\
  % \And
  % Coauthor \\
  % Affiliation \\
  % Address \\
  % \texttt{email} \\
  % \And
  % Coauthor \\
  % Affiliation \\
  % Address \\
  % \texttt{email} \\
}

\begin{document}

\maketitle

\begin{abstract}
Protein structure representation learning has emerged as a critical area in machine learning (ML) and computational biology, with promising applications in drug discovery, protein design, and protein function prediction. Despite promising advancement, there remains a need for a robust, standardized benchmark to measure and compare the performance of various representation learning algorithms with greater granularity and relevance to downstream applications. In this work, we introduce a comprehensive and open benchmark suite for evaluating protein structure representation learning methods.

We provide several pre-training corpuses comprised of both experimental and predicted structures, offering a balanced challenge to representation learning algorithms. We also propose a set of performance metrics, including classification accuracy, retrieval performance, and transfer learning capabilities, to assess various aspects of the learned representations. These metrics enable the evaluation of the quality of the embeddings, the structural and functional relationships captured, and their usefulness in downstream tasks such as protein-protein interaction prediction and protein folding. We evaluate several state-of-the-art geometric graph neural network structural encoders, including. Our analysis reveals the strengths and weaknesses of each method, highlighting potential avenues for improvement and guiding future research in the field.

We aim to establish a common ground for the machine learning and computational biology communities to collaborate, compare, and advance protein structure representation learning research. By providing a standardized evaluation platform, we expect to accelerate the development of novel methodologies and improve our understanding of protein structures and their functions. The benchmark incorporates several engineering contributions which we believe will reduce the barrier to pre-training and working with large structure-based models.
\end{abstract}

\section{Introduction}
Proteins play central roles in cellular processes and understanding their function is...


Developments in protein structure prediction methods have led to an explosion in the availability of structural data. This has resulted in a significant gap between meaningful annotations due to the significant. 


Several deep learning methods have been developed to learn from protein structures. 


In this work we develop a large-scale framework for developing and evaluating protein structural encoders, providing several pre-training corpuses that span known foldspace and tasks that assess the ability of models to learn informative representations at different levels of structural granularity.

Previous works in protein structure representation learning have focussed on learning effective \emph{global} (i.e. graph-level) representations of protein structure, typically evaluating the methods on function or fold classification tasks. However, there has been comparatively little investigation into the ability of different methods to learn informative local (\emph{node-level}) representations. Good node-level representations are important for a variety of annotation tasks, such as binding or interaction site prediction, as well as providing conditioning signals in structure-conditioned molecule design methods. Crucially, understanding the structure-function relationship at this granular level can drive progress in protein design by revealing structural motifs that underlie desirable properties enabling them to be incorporated into designs.



Our contributions are as follows:

\begin{enumerate}
    \item 
\end{enumerate}
\section{Related Work}

\paragraph{Protein Structure Representation Learning}




\paragraph{Protein Benchmarks} Several benchmarks have been proposed for evaluating the efficacy of learnt protein \emph{sequence} representations. However, \emph{structure-based} benchmarks are comparatively unaddressed. \citet{tape} developed the TAPE (Tasks assessing protein embeddings) benchmark, providing a large pre-training corpus of protein sequences curated from Pfam \cite{ElGebali2018}, as well as a collection of five supervised benchmark tasks assessing the ability of protein language models to predict structural qualities (contact prediction and secondary structure prediction), and functional properties (fluorescence and stability prediction). \citet{peer} developed the PEER (Protein Sequence Understanding) benchmark, focussing on multitask evaluation of protein sequence models. 

\begin{enumerate}
    \item X et al. developed FLIP
    \item X et al developed Atom3D, the first benchmark designed around structure-based tasks.
    \item TDC provides several datasets..
    \item TorchProtein
\end{enumerate}



\paragraph{Denoising-based pre-training and Regularisation} 
Several methods have been developed for pre-training GNNs, predominantly focussing on cases where 3D coordinate information is only implicitly encoded in the graph structures. In this work, we build on work by \citet{godwin2021simple} and \citet{zaidi2023pretraining} to investigate whether denoising-based auxillary and pre-training tasks are effective methods for pre-training geometric GNNs operating on protein structures.
\section{Benchmark}
The overarching goal of the benchmark is to effectively cover the design space of protein structure representation learning methods. To achieve this, the benchmark is highly modular by design, enabling evaluation of different combinations of structural encoders, protein featurisation schemes, and auxiliary tasks over a wide range of both supervised and unsupervised tasks.

\subsection{Featurisation Schemes}
Protein structures are typically represented as graphs, with researchers typically opting to use C$\alpha$ atoms as nodes as full atom representations can quickly become computationally intractable due to the explosion in graph size. However, this is a lossy representation, with much of the structural detail, such as backbone and sidechain structure, being only implicitly encoded. Due to the computational burden incurred by operating on full-atom node representations, we focus primarily on C$\alpha$ based-graph representations, investigating featurisation strategies to incorporate higher-level structural information though we do provide utilities to enable users to work with backbone and full-atom graphs.



\begin{table*}[!ht]
    \centering
    \caption{Overview of supervised tasks and datasets}
    \begin{adjustbox}{max width=\linewidth}
        \begin{tabular}{lcccccc}
        \toprule
        \textbf{Task} & \textbf{Category} & \textbf{Dataset Origin} & \textbf{Structures} &  \textbf{\# Train} & \textbf{\# Validation} & \textbf{\# Test} \\
        \hline
        Gene Ontology Prediction & Protein classification & \citet{Gligorijevi2021} & Experimental \\
        Fold Prediction & Protein classification & Houe et al. & Experimental
        \end{tabular}
    \end{adjustbox}
    \vspace{-1em}
\end{table*}

\subsection{Pre-training Tasks}
We provide a comprehensive suite of pre-training tasks. Broadly, these tasks can be categorised into: masked-attribute prediction, denoising-based and contrastive learning based tasks. Each of these tasks can be used as a primary training objective, e.g. in a pre-training set-up or as auxiliary tasks in a downstream supervised task.

\subsubsection{Denoising Tasks}


\paragraph{Sequence Denoising} The benchmark contains two sequence denoising variations. The first is based on mutating a fraction of the residues to a random amino acid and tasking the model with recovering the uncorrupted sequence. The second is a masked residue prediction task, where a fraction of the residues are altered to a mask value and the model is tasked to recover the uncorrupted sequence.

\paragraph{Structure Denoising} We provide two structure-based denoising tasks: coordinate denoising and torsional denoising. In the coordinate denoising task, noise is sampled from a normal or uniform distribution and scaled by noise factor, $\nu \in \Re^3$, and applied to each of the atom coordinates in the structure to ensure structural features, such as backbone or sidechain torsion angles, are also corrupted. The model can then be tasked with predicting either the per-node noise or the original uncorrupted coordinates. For the torsional denoising variant, the noise is applied to the backbone torsion angles and the cartesian coordinates are recomputed using pNeRF \cite{AlQuraishi2019} prior to feature computation. Similarly to the coordinate denoising task, the model can then be tasked with predicting either the per-residue angular noise or the original dihedral angles.

\paragraph{Sequence-Structure Co-Denoising} This task is a multitask formulation of the previously described structure and sequence denoising tasks, where separate output heads are used to denoise each of the modalities.

\subsubsection{Masked Attribute Prediction Tasks}


\paragraph{Edge Distance Prediction}


\paragraph{Angle Prediction}

\paragraph{Dihedral Angle Prediction}


\subsubsection{Self-supervised Tasks}
\paragraph{pLDDT Prediction}
Structure prediction models typically provide per-residue pLDDT (predicted Local Distance Difference Test) scores as local confidence measures in the quality of the prediction. We formulate a self-supervision task on predicted structures, somewhat analagous to structure quality assessment (QA), where the model is tasked with predicting the scaled residue-wise pLDDT $y \in [0, 1]$ values.



\subsection{Supervised Tasks}
We curate several structure-based and sequence-based datasets from the literature\footnote{To retain focus on \emph{protein} representation learning, we deliberately exclude commonly-used tasks based on protein-small molecule interactions as it is hard to disentangle the effect of the small molecule representation and the potential for bias \cite{Boyles2019}}. Importantly, the raw structures are retrieved directly from the PDB enabling users to provide a custom sequence of pre-processing steps, such as deprotonation or fixing missing regions which are common in experimental data.

\subsubsection{Node-level Tasks}
\paragraph{Paratope Prediction}
\paragraph{PPI Site Prediction}
\paragraph{Metal Binding Site Prediction}

\subsubsection{Graph-level Tasks}
\paragraph{Gene Ontology Prediction}
\paragraph{Remote Homology Detection}
\paragraph{Protein Structure Ranking}
\paragraph{Antibody Developability Prediction}








\subsection{Pre-training Datasets}
The benchmark contains several large corpuses of both experimental and predicted structural data that can be used for pre-training or inference.

\subsubsection{Experimental Structures}
\paragraph{PDB} We provide utilities for curating datasets directly from the Protein Data Bank \cite{Berman2000}. In addition to using the collection in its entirety, users can provide custom filters to subset and split the data using a combination of structural similarity, sequence similarity or temporal strategies. Structures can be filtered by length, number of chains, resolution, deposition date, presence/absence of particular ligands and structure determination method. We provide support for working with PDB structures in both \texttt{.pdb} and \texttt{.mmtf} format \cite{Bradley2017}, which significantly reduces the disk-space requirements for storing the data.


\paragraph{CATH} We employ the non-redundant dataset splits developed by Ingraham et al. as an additional, smaller pre-training dataset.




\subsubsection{Predicted Structures}
We provide ready-to-go dataloaders for several collections of predicted structures derived from both AlphaFold2 and ESMFold. By developing a dataloaders that operate on FoldComp databases, a modestly lossy compression scheme for predicted protein structures, we are able to achieve significant reduction in disk-space requirements with no discernible performance degradation. We believe this will significantly benefit the field.







\section{Methods and Experimental Setup}
\begin{itemize}
    \item Paragraph justifying our choices of a subset of tasks and models
\end{itemize}


\paragraph{Architectures}
We evaluate X geometric GNN architectures, spanning the range of message passing body order and tensor order. 


\paragraph{Pre-training Dataset} For all pre-training tasks we use the \texttt{afdb\_rep\_v4} dataset compiled by \citet{BarrioHernandez2023}. This dataset contains 2.27 million representative structures, identified through large-scale structural-similarity based clustering of the 214 million structures contained in the AlphaFold Database \cite{Varadi2021} using FoldSeek \cite{vanKempen2023}. This dataset therefore provides a rich diversity of protein structures and is substantially larger than any other previously used structure-based pre-training corpus that we are aware of, whilst remaining of a size that is amenable to experimentation.

\paragraph{Featurisation Schemes}
The benchmark includes comprehensive featurisation schemes for both scalar and vector-valued feature computation. 


\paragraph{Noising Schemes} For structure-based denoising we draw noise samples from a guassian distribution and scale by 0.1. For structure-based denoising, we use the mutation strategy and corrupt 25\% of the residues in each protein. When denoising is used as an auxillary task, we weight the loss with a coefficient $\lambda = 0.1$, similar to NoisyNodes \cite{godwin2021simple}.

\paragraph{Training}
We use a ReduceLRonPlateau learning rate scheduler for the downstream tasks with a patience of 5 epochs and a reduction factor of 0.6. For the pre-training tasks, we use a linear warmup with cosine decay schedule.



\subsection{Baselines}
\section{Results}


\subsection{Baseline Results on Graph Classification Tasks}


\subsection{Graph Classification tasks with denoising auxiliary tasks}


\subsection{Pre-training via denoising}



\section{Future Work}
%Protein representation learning is an exciting field with incredible room for expansion, innovation, and impact. The exponentially growing gap between labeled and unlabeled protein data means that self-supervised learning will continue to play a large role in the future of computational protein modeling. Our results show that no single self-supervised model performs best across all protein tasks. We believe this is a clear challenge for further research in self-supervised learning, as there is a huge space of model architecture, training procedures, and unsupervised task choices left to explore. It may be that language modelling as a task is not enough, and that protein-specific tasks are necessary to push performance past state of the art. Further exploring the relationship between alignment-based and learned representations will be necessary to capitalize on the advantages of each. We hope that the datasets and benchmarks in TAPE will provide a systematic model-evaluation framework that allows more machine learning researchers to contribute to this field.


\section{Societal Impact}
This work focuses on building a comprehensive and multi-task benchmark for protein structure representation learning. In this benchmark, we provide several large pre-training corpuses, featurisation schemes, model implementations and benchmarking tasks to evaluate the effectiveness of protein sequence encoding methods. The variety of tasks can enable us to develop insight into effective pre-training strategies, and whether pre-trained protein structural representations can have material impact in real-world computational biology and drug discovery research activities. It is not lost on us that these models can play a role in developing, for example, harmful chemical matter in the hands of a bad actor. Additionally, the training and pre-training of very large models can contribute to climate change. However, we believe the developing highly effective structural representations will have broad, positive implications across biology and medicine that significantly outweigh the potential for misuse.

%\bibliography
\bibliography{bibliography}



%%%%%%%%%%%%%%%%%%%%%%%%%%%%%%%%%%%%%%%%%%%%%%%%%%%%%%%%%%%%
%\section*{Checklist}

%%% BEGIN INSTRUCTIONS %%%
The checklist follows the references.  Please
read the checklist guidelines carefully for information on how to answer these
questions.  For each question, change the default \answerTODO{} to \answerYes{},
\answerNo{}, or \answerNA{}.  You are strongly encouraged to include a {\bf
justification to your answer}, either by referencing the appropriate section of
your paper or providing a brief inline description.  For example:
\begin{itemize}
  \item Did you include the license to the code and datasets? \answerYes{See Section~\ref{gen_inst}.}
  \item Did you include the license to the code and datasets? \answerNo{The code and the data are proprietary.}
  \item Did you include the license to the code and datasets? \answerNA{}
\end{itemize}
Please do not modify the questions and only use the provided macros for your
answers.  Note that the Checklist section does not count towards the page
limit.  In your paper, please delete this instructions block and only keep the
Checklist section heading above along with the questions/answers below.
%%% END INSTRUCTIONS %%%

\begin{enumerate}

\item For all authors...
\begin{enumerate}
  \item Do the main claims made in the abstract and introduction accurately reflect the paper's contributions and scope?
    \answerTODO{}
  \item Did you describe the limitations of your work?
    \answerTODO{}
  \item Did you discuss any potential negative societal impacts of your work?
    \answerTODO{}
  \item Have you read the ethics review guidelines and ensured that your paper conforms to them?
    \answerTODO{}
\end{enumerate}

\item If you are including theoretical results...
\begin{enumerate}
  \item Did you state the full set of assumptions of all theoretical results?
    \answerTODO{}
	\item Did you include complete proofs of all theoretical results?
    \answerTODO{}
\end{enumerate}

\item If you ran experiments (e.g. for benchmarks)...
\begin{enumerate}
  \item Did you include the code, data, and instructions needed to reproduce the main experimental results (either in the supplemental material or as a URL)?
    \answerTODO{}
  \item Did you specify all the training details (e.g., data splits, hyperparameters, how they were chosen)?
    \answerTODO{}
	\item Did you report error bars (e.g., with respect to the random seed after running experiments multiple times)?
    \answerTODO{}
	\item Did you include the total amount of compute and the type of resources used (e.g., type of GPUs, internal cluster, or cloud provider)?
    \answerTODO{}
\end{enumerate}

\item If you are using existing assets (e.g., code, data, models) or curating/releasing new assets...
\begin{enumerate}
  \item If your work uses existing assets, did you cite the creators?
    \answerTODO{}
  \item Did you mention the license of the assets?
    \answerTODO{}
  \item Did you include any new assets either in the supplemental material or as a URL?
    \answerTODO{}
  \item Did you discuss whether and how consent was obtained from people whose data you're using/curating?
    \answerTODO{}
  \item Did you discuss whether the data you are using/curating contains personally identifiable information or offensive content?
    \answerTODO{}
\end{enumerate}

\item If you used crowdsourcing or conducted research with human subjects...
\begin{enumerate}
  \item Did you include the full text of instructions given to participants and screenshots, if applicable?
    \answerTODO{}
  \item Did you describe any potential participant risks, with links to Institutional Review Board (IRB) approvals, if applicable?
    \answerTODO{}
  \item Did you include the estimated hourly wage paid to participants and the total amount spent on participant compensation?
    \answerTODO{}
\end{enumerate}

\end{enumerate}

%%%%%%%%%%%%%%%%%%%%%%%%%%%%%%%%%%%%%%%%%%%%%%%%%%%%%%%%%%%%

\appendix

\section{Appendix}

Include extra information in the appendix. This section will often be part of the supplemental material. Please see the call on the NeurIPS website for links to additional guides on dataset publication.

\begin{enumerate}

\item Submission introducing new datasets must include the following in the supplementary materials:
\begin{enumerate}
  \item Dataset documentation and intended uses. Recommended documentation frameworks include datasheets for datasets, dataset nutrition labels, data statements for NLP, and accountability frameworks.
  \item URL to website/platform where the dataset/benchmark can be viewed and downloaded by the reviewers.
  \item Author statement that they bear all responsibility in case of violation of rights, etc., and confirmation of the data license.
  \item Hosting, licensing, and maintenance plan. The choice of hosting platform is yours, as long as you ensure access to the data (possibly through a curated interface) and will provide the necessary maintenance.
\end{enumerate}

\item To ensure accessibility, the supplementary materials for datasets must include the following:
\begin{enumerate}
  \item Links to access the dataset and its metadata. This can be hidden upon submission if the dataset is not yet publicly available but must be added in the camera-ready version. In select cases, e.g when the data can only be released at a later date, this can be added afterward. Simulation environments should link to (open source) code repositories.
  \item The dataset itself should ideally use an open and widely used data format. Provide a detailed explanation on how the dataset can be read. For simulation environments, use existing frameworks or explain how they can be used.
  \item Long-term preservation: It must be clear that the dataset will be available for a long time, either by uploading to a data repository or by explaining how the authors themselves will ensure this.
  \item Explicit license: Authors must choose a license, ideally a CC license for datasets, or an open source license for code (e.g. RL environments).
  \item Add structured metadata to a dataset's meta-data page using Web standards (like schema.org and DCAT): This allows it to be discovered and organized by anyone. If you use an existing data repository, this is often done automatically.
  \item Highly recommended: a persistent dereferenceable identifier (e.g. a DOI minted by a data repository or a prefix on identifiers.org) for datasets, or a code repository (e.g. GitHub, GitLab,...) for code. If this is not possible or useful, please explain why.
\end{enumerate}

\item For benchmarks, the supplementary materials must ensure that all results are easily reproducible. Where possible, use a reproducibility framework such as the ML reproducibility checklist, or otherwise guarantee that all results can be easily reproduced, i.e. all necessary datasets, code, and evaluation procedures must be accessible and documented.

\item For papers introducing best practices in creating or curating datasets and benchmarks, the above supplementary materials are not required.
\end{enumerate}


%%%%%%%%%%%%%%%%%%%%%%%%%%%%%%%%%%%%%%%%%%%%%%%%%%%%%%%%%%%%
\end{document}
